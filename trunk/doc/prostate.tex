\documentclass[10pt]{article}

\usepackage{amsmath}
\usepackage{bm}
\usepackage{hyperref}
\usepackage[boxed]{algorithm2e}
\usepackage{xcolor}

\usepackage[framemethod=TikZ]{mdframed}

\newcounter{pnote}
\newenvironment{aside}[0]
  {\addtocounter{pnote}{1}
    \begin{center}
    \mdfsetup{tikzsetting={draw=black,dashed,line width=1pt,dash pattern = on 5pt off 3pt},linecolor=white,outerlinewidth=1pt}
    \begin{mdframed}
    \noindent \textsc{Aside}\#\arabic{pnote}:\newline}
  {\end{mdframed}\end{center}}

\title{A Biomechanically Constrained Point Drift Algorithm \\ {\large Notes and Modifications}}
\author{Siavash Khallaghi, C.~Antonio S\'anchez}

\newcommand{\di}[2]{\frac{\partial#1}{\partial#2}}
\newcommand{\trans}[1]{#1^{T}}
\newcommand{\trace}{\mathrm{tr}}
\newcommand{\diag}{\mathrm{diag}}
\newcommand{\kron}{\mathrm{kron}}

\newcommand{\uvec}{\bm{u}}
\newcommand{\xvec}{\bm{x}}
\newcommand{\yvec}{\bm{y}}
\newcommand{\vvec}{\bm{v}}

\begin{document}
 
\maketitle

\section{Introduction}
I would change this sentence:
\begin{quotation}
  ``Our experience shows significant change in prostate shape during such freehand scan-
ning, well beyond [the 10\% that] has been observed...'' 
\end{quotation}
For what follows regarding approximations in the stiffness matrix, we actually want 
evidence that the deformations are relatively small.  Maybe you can call the observed large
difference in shape as the ``apparent'' deformation, which is a combination of prostate 
motion, deformation, and the difficulties in reconstructing a 3D volume from 2D slices
taken over a finite period of time.

Following up with this statement:
\begin{quotation}
  I chose the group-wise point set SSM generation model by Rasoulian et al. [7] because of
  ...
\end{quotation}
From what I took from the paper, I would say the group-wise method was specifically designed 
for situations where we either don't know or don't have explicit point correspondences between
samples in the training data for the SSM.  If this is the case for your training data, 
which I believe is the case, then this is a very good reason to use this method.
For instance, if each training shape is just a point cloud on the surface of interest, then 
the points in one set will not in general correspond to points in another set.  The only
time we do have known correspondences is if the points in the data set correspond to known
features, or if they come from some type of tagged dynamic imaging on a single subject and we can
deduce which tags match between frames.  Otherwise, we need to estimate correspondences, either
by ICP, or the group-wise method uses a generalized from of CPD which will better handle
noise, outliers, and missing data.


\section{Method}

\setcounter{subsection}{-1} %decrease for added section
\subsection{A Note about FEM and Stiffness Matrices}

For static FEM, stiffness matrices are derived by minimizing the total energy of a model:
\begin{align}
  E(\Omega) & = \int_\Omega W(x) dx,
\end{align}
where $W=\sigma^T\epsilon$ represents the elastic energy for linear materials.  If there are
external body forces and dirichlet (fixed) boundary conditions, then you want to minimize
\begin{align}
  E(\Omega) & = \int_\Omega \sigma^T(x)\epsilon(x) dx + \int_\Omega u^T(x)f(x) dx, \label{eq:energy}\\
  \text{subject to } \left.u(x)\right|_{\partial \Omega} & = b(x)\notag
\end{align}
In FEM, we discretize the displacement over the volume as
\begin{align}
  u(x) &= \sum_{i=0}^N\phi_i(x)u_i, \label{eq:discretize}
\end{align}
where the shape functions $\{\phi_i\}$ form some partition of unity over the space:
\begin{align}
 \sum_{i=0}^N\phi_i(x) = 1, \quad \forall x\in\Omega.
\end{align}
The set $\{u_i\}$ are
the degrees of freedom in the model.  Each distinct region where a set of the shape functions overlap 
is called an \emph{element}.  Furthermore, the shape functions are typically defined such at a set of points 
point $\{x_i\}$,
\begin{align}
  \phi_k(x_j)& =  \begin{cases}
		    1, & j=k\\
		    0, & j\neq k
		  \end{cases}
\end{align}
We call these points $\{x_i\}$ \emph{nodes}.  In such a case, we have $u(x_i)=u_i$, so it turns out 
that $\{u_i\}$ are the displacements at the nodes. 

Substituting equation \eqref{eq:discretize} into an expression for strain, we can write
\begin{align}
  \epsilon(x) & = \sum_{i=1}^N B_i(x) u_i,\\
  B_i & = \begin{bmatrix}
	  \di{\phi_i}{x_1} & 0 & 0\\
	  0 & \di{\phi_i}{x_2} & 0\\
	  0 & 0 & \di{\phi_i}{x_3}\\
	  \di{\phi_i}{x_2} & \di{\phi_i}{x_1} & 0\\
	  \di{\phi_i}{x_3} & 0 & \di{\phi_i}{x_1}\\
	  0 & \di{\phi_i}{x_3} & \di{\phi_i}{x_2}
        \end{bmatrix}
\end{align}
Important note: here we are using the expression for strain, following \cite{bonet:2000:fem}:
\begin{align}
  \epsilon^T(x) & = \begin{bmatrix}
                   \di{u_1}{x_1}, &  \di{u_2}{x_2}, & \di{u_3}{x_3}, 
                   & \di{u_1}{x_2}+\di{u_2}{x_1}, & \di{u_1}{x_3}+\di{u_3}{x_1}, 
                   & \di{u_2}{x_3}+\di{u_3}{x_2}
                  \end{bmatrix} \notag
\end{align}
In some texts, the first three terms have a factor of 2 included.  This is compensated
for in the stress-strain relationship.  Also, this expression for strain
is only valid for small linear-strains.  It breaks down for large strains or rotations.

For linear FEM, there is a linear stress-strain relationship assumed:
\begin{align}
  \sigma(x) & = D\epsilon(x),\\
  D & = \begin{bmatrix}
	  \lambda +2\mu & \lambda & \lambda  & 0 & 0 & 0\\
	  \lambda &  \lambda +2\mu & \lambda & 0 & 0 & 0\\
	  \lambda & \lambda & \lambda +2\mu & 0 & 0 & 0\\
	  0 & 0 & 0 & \mu & 0 & 0\\
	  0 & 0 & 0 & 0 & \mu & 0\\
	  0 & 0 & 0 & 0 & 0 & \mu
        \end{bmatrix}\\
  \lambda & = \dfrac{E\nu}{(1+\nu)(1-2\nu)} \notag\\
  \mu & = \dfrac{E}{2(1+\nu)}, \notag
\end{align}
where $E$ is the Young's Modulus, and $\nu$ is Poisson's ratio.  Substituting
all these expressions into Equation \eqref{eq:energy}, we arrive at the system:
\begin{align}
   E(\Omega) & = \sum_{i=1}^N\sum_{j=1}^N u_j^T \left[\int_\Omega B_j^TDB_i dx\right]\, u_i + \int_\Omega \phi_i(x)f^T(x) dx\,u_i. \label{eq:discreteenergy}
\end{align}
Equation \eqref{eq:discreteenergy} can be compared to Equation (3) of Marami \textit{et al.}~\cite{marami:2011:femreg}, noting a few things.
\begin{itemize}
  \item By $E(u_i)$, they are really defining:
  \begin{align}
    E_i(u_i) & = \sum_{j=1}^N u_j^T \left[\int_\Omega B_j^TDB_i dx\right]\, u_i + \int_\Omega \phi_i(x)f^T(x) dx\,u_i,
  \end{align}
  so that $E=\sum_i E_i$. They aren't really evaluating the energy at a single point.  
  \item  The superscript $el$ is indicating that they are restricting themselves to a single element at a time.
  \item For tetrahedral elements and a given node $u_i$, the shape function $\phi_i$ only overlaps with
  four other shape functions, hence $j=1\ldots4$.
  \item They define $f_i(x)$ to be $\phi_i(x)f(x)$
  \item These are scalar functions, so I've switched the transposes so that $u_i$ appears at the end.
\end{itemize}
Important note: these integrals are in terms of the absolute coordinate system, so are over the deformed state of the object.  
In general, we do not have expressions for the deformed shape functions, $\phi_i$.  We must do a change of variables so we
can integrate over the `rest shape'.  This introduces a Jacobian:
\begin{align}
  E(\Omega) & = \sum_{i=1}^N\sum_{j=1}^N u_j^T \left[\int_{\Omega_0} B_j^TDB_i \left|\di{x}{X}\right| dX\right]\, u_i + \int_{\Omega_0} \phi_i(X)f(X) \left|\di{x}{X}\right|dX\,u_i.
  \label{eq:strainenergy}
\end{align}
When attempting to minimize energy, the stiffness matrix becomes:
\begin{align}
 K_{i,j} & = \left[\int_{\Omega_0} B_j^TDB_i \left|\di{x}{X}\right| dX\right], \label{eq:stiffness}
\end{align}
which depends on the current \emph{deformed} state because of the Jacobian term.  Thus, the stiffness matrix evolves over time in 
a dynamic simulation.  If we \emph{again} assume small deformation, we might approximate $\left|\di{x}{X}\right|\approx I$, which would
give a fixed stiffness matrix.

For a more in-depth discussion of FEM, see \cite{bonet:2000:fem}.
 
\subsection{General Framework}

We use the following notation:
\smallskip

\noindent\begin{tabular}{lp{0.7\textwidth}}
  \hline
  $D$ & dimension of the point sets\\
  $N, M$ & number of points in the target \emph{surface geometry} and SSM, respectively\\
  $X_{N\times D}$ & $(x_1,\ldots,x_N)^T$, the tracked 2D-TRUS segmented points\\
  $Z_{M\times D}$ & $(z_1,\ldots,z_M)^T$, the GMM centroids for the mean shape of the SSM\\
  $Y_{M\times D}$ & $(y_1,\ldots,y_M)^T$, the GMM centroids of an \emph{instance} of the SSM\\
  $\Psi$ & matrix of deformation modes, such that $Y=Z+\Psi b$\\
  $V_{M\times D}$ & $(v_1,\ldots,v_M)^T$, the \emph{deformed} portion of the displacement of the \emph{instance} shape\\
  $T(Z,\theta)$ & the transform applied to $Z$, where $\theta$ is a set of transformation parameters\\
  \hline
  $J$ & the number of FEM nodes used to regularize the displacement field\\
  $P_{J\times D}$ & $(p_1,\ldots,p_J)^T$, the FEM node locations\\
  $\Phi$ & the linear interpolation matrix used to embed the SMM instance $Y$ in the FEM model\\
  \hline
  $\xvec_{DN\times 1}$ & $(\trans{x}_1,\ldots,\trans{x}_N)^T$\\
  $\yvec$ & $(\trans{y}_1,\ldots,\trans{y}_N)^T$\\
  $\uvec$ & $(\trans{u}_1,\ldots,\trans{u}_N)^T$\\
  $\vvec$ & $(\trans{v}_1,\ldots,\trans{v}_N)^T$\\
\end{tabular}

$\vdots$

The transformation can be written as a chain:
\begin{align}
  T(Z,\theta) = T_\mathrm{rigid}\left(T_\mathrm{fem}\left(T_\mathrm{ssm}(Z)\right)\right)
\end{align}
where $T_\mathrm{ssm}(\cdot)$ denotes the transformation from the mean shape to an instance shape ($Z\to Y$) using a linear
combination of SSM modes. $T_\mathrm{fem}$ is the deformation of the instance shape.  $T_\mathrm{rigid}$ is the rigid transform,
consisting of a rotation followed by a translation.

Note: it is important to have the deformation inside the rigid transform so that the global translation and rotation are removed.  Otherwise, the small
strain assumption will break down.  It also conceptually makes more sense, since you can imagine the prostate deforming
independent of its current global location.  The deformation is then part of its `shape'.

Following the derivation for CPD \cite{myronenko:2010:cpd}, we can write an objective function
\begin{align}
 Q(R,t,b,V) & = \dfrac{1}{2\sigma^2}\sum_{m,n=1}^{M,N} P(z_m|x_n)\|x_n-R(z_m+b\Psi_m+v_m)- t\|^2 \notag\\ 
  & \qquad + \dfrac{N_PD}{2}\log(\sigma^2) + \mathrm{Reg}(b,V),
\end{align}
where $N_P=\sum_{m,n=1}^{M,N}P(z_m|x_n)$.  The regularization term on shape parameters and deformation can be decoupled:
\begin{align}
  \mathrm{Reg}(b,V) & = \dfrac{\mu}{2}b^T\Lambda b + \dfrac{\lambda}{2}\mathrm{Reg}(V).
\end{align}

\subsection{Proposed Solution}

\begin{algorithm}[H]
 \SetAlgoLined
 \KwData{Require $Z$, $\Psi$, $X$}
 Initialize: $R$, $t$, $b$, $v$\;
 \While{not converged}{
  Rigid registration betwen $X$ and $Y+V$\;
  Update $R$, $t$\;
  Shape registration between $X$ and $R(Z+b\Psi+V)+t$\;
  Update $b$ which updates $Y$\;
  Deformable registration between $X$ and $R(Y+V)+t$\;
  Update $V$\;
 }
 \caption{Pseudo code representation of registration framework \label{alg:registration}}
\end{algorithm}

\subsection{Biomechanically Constrained Non-rigid Registration}

Following Algorithm \ref{alg:registration} and ignoring the terms that do not depend on $V$, we
can re-write the objective function as:
\begin{align}
  Q(V) & = \dfrac{1}{2\sigma^2}\sum_{m,n=1}^{M,N}\left\|x_n - R\left(y_m+v_m\right)-t\right\|^2 + \mathrm{Reg}(V).
\end{align}
In order to regularize the displacement parameters, $V$, we follow \cite{marami:2011:femreg} and introduce a finite element framework.
Until now, we have only dealt with surface shape data.  The SSMs are defined based on segmented surface, not the interior.  Thus, 
we need to embed this surface into a finite element model.  There are three options for this:
\begin{enumerate}
 \item Create a volumetric mesh from the surface mesh \emph{without} introducing additional nodes in the interior. \label{itm:without}
 \item Create a volumetric mesh from the surface mesh \emph{with} added nodes in the interior.  \label{itm:with}
 \item Create a volumetric mesh that encapsulates the surface mesh, and couple the two using the FEM interpolation field. \label{itm:embedded}
\end{enumerate}
With method \#\ref{itm:without}, the number of FEM nodes will equal the number of vertices of the surface mesh/SSM model, with a one-to-one mapping:
\begin{align}
  Y = I P,
\end{align}
where $P=(p_1,\ldots,p_M)^T$ are the FEM node locations at rest, and $Y$ are the SSM instance vertex locations, and $I$ is the identity matrix.  
However, it is not always possible to triangulate a polyhedron without adding additional vertices \cite{schonhardt:1928:tet,toussaint:1993:tetrahedralization}.  
Futhermore, the elements will be poorly conditioned (highly skewed, elongated), which affects their accuracy and stability.

With method \#\ref{itm:with}, we can automatically generate a well-conditioned tetrahedral mesh of the prostate using Delauney Triangulation 
\cite{si:2008:tetgen}. Each vertex in the original surface mesh will correspond to an FEM node, but there will be additional nodes
added to the interior of the shape:
\begin{align}
  Y = \begin{bmatrix}
	I_{M\times M} & 0
      \end{bmatrix} P,
\end{align}
where $P$ now has size $J>M$.  Note that in this case, we \emph{cannot} recover node locations by only knowing the surface vertex positions.

For method \#\ref{itm:embedded}, no direct link between node location and SSM vertex is required.  Start by distributing nodes in such a way
that the corresponding finite element model encapsulates all SSM vertex locations.  One way this can be accomplished is to determine a bounding
box for the surface in question, then distribute a regular grid of FEM nodes to create a basic hexahedral encapsulating model.  The space
can then be discretized based on the nodes:
\begin{align}
  x(X) & = \sum_{i=1}^J \phi_i(X)p_i,
\end{align}
where $X$ is the reference position in space.  This lets us write the SSM instance in terms of node positions as:
\begin{align}
  y_m & = x(y_m) = \sum_{i=1}^J \phi_i(y_m)p_i.
\end{align}
Writing this in matrix notation, we have:
\begin{align}
 Y & = \Phi P, \label{eq:interpolation}
\end{align}
where $\Phi$ is an $M\times J$ interpolation matrix.  Note that both previous methods are also captured
in this framework by taking $\Phi=I$ and $\Phi=[I\;0]$ respectively.  When the FEM model deforms, each node is displaced by
some amount:
\begin{align}
  p_i & = p_{i,0} + u_i,
\end{align}
where $p_i{i,0}$ is the rest position of node $i$.  We can then recover the deformed SSM shape as:
\begin{align}
  (Y+V) & = \Phi(P_0+U), \notag \\
  \quad \implies V & = \Phi U, \label{eq:uV}
\end{align}
where $U=(u_1,\ldots,u_J)^T$ are displacements of the FEM nodes.  Note that for this method, every time the SSM is updated,
we have to recompute $\Phi$ to reflect the new undeformed state.

In \cite{marami:2011:femreg}, Marami \textit{et al.} use the FEM elastic energy as a regularization term in their
objective function:
\begin{align}
   \mathrm{Reg}(V) & = \dfrac{\lambda}{2}\trans{\uvec}K\uvec,
\end{align}
where $K$ is the global stiffness matrix.  In our case, however, there is this distinction between FEM nodes and vertex locations, 
given by the relationship in Equation \eqref{eq:interpolation}.

\subsubsection{Equivalent Stiffness Matrix}

We can re-write the energy function in Equation \eqref{eq:discreteenergy} as:
\begin{align}
  E(\Omega) & = \sum_{i=1}^N\sum_{j=1}^N u_j^T K_{i,j}\, u_i + f_i^T\,u_i \notag\\
  & = \sum_{j=1}^N u_j^T K_j \uvec + f^T \uvec, \notag\\
  & = \uvec^TK\uvec + f^T\uvec, \label{eq:matrixenergy}
\end{align}
where $K$ is the global stiffness matrix.

If we could create an `equivalent' stiffness matrix that only included vertex locations, we could then follow the same procedure as Marami 
\textit{et al.}~\cite{marami:2011:femreg}. To do so, we must replace each instance of $u$ in Equation \eqref{eq:matrixenergy} with an 
expression involving only $V$.  For the moment, let us assume that we can solve the system in Equation \eqref{eq:uV} for $U$ in terms of 
$V$, at least in a least-squares sense:
\begin{align}
  U = (\Phi^T\Phi)^{-1}\Phi^T V.
\end{align}
Substituting this back into Equation \eqref{eq:matrixenergy}, we have:
\begin{align}
  E(\Omega) & = V^T\Phi(\Phi^T\Phi)^{-1} K (\Phi^T\Phi)^{-1}\Phi^T V + f^T (\Phi^T\Phi)^{-1}\Phi^T V.
\end{align}
Thus, the `equivalent' stiffness matrix is given by 
\begin{align}
  K_V & = \Phi(\Phi^T\Phi)^{-1} K (\Phi^T\Phi)^{-1}\Phi^T.
\end{align}

Unfortunately, the matrix $\Phi^T\Phi$ will not in general be invertible.  For example, taking Method \#\ref{itm:with}, we have
\begin{align}
  \Phi^T\Phi & = \begin{bmatrix}
                  I_{M\times M} & 0\\
                  0 & 0
                 \end{bmatrix}
\end{align}
It turns out there will be many such cases. If ever there are more nodes than vertices, or elements that contain no point of $Y$,
then $\Phi^T\Phi$ will not be invertible.  One option may be to simply take the pseudo-inverse, however this seems less than ideal.

\subsubsection{Mixed Formulation}

In general, neither $\Phi$ nor $\Phi^T\Phi$ will be invertible, so we cannot recover node positions knowing only the SSM instance.
Thus, we need to introduce the FEM nodes as additional degrees of freedom in our optimization.  They are treated as parameters of
the deformable transform.  Deformation is regularized by minimizing strain energy of the FEM node positions.  These positions
are then also coupled to the SSM vertices through the interpolation given in Equation \eqref{eq:uV}:
\begin{align}
  Q(U,V,s, R,t) & = \dfrac{1}{2\sigma^2}\sum_{m,n=1}^{M,N}P(y_m|x_n)\left\|x_n- R\left(y_m+v_m\right)-t\right\|^2 \notag\\
  & \qquad + \dfrac{N_PD}{2}\log(\sigma^2) + \mathrm{Reg}(U,V)\\
  \mathrm{Reg}(U,V) & = \dfrac{\alpha}{2}\sum_{i=1}^M\|\trans{v}_i-\Phi_iU\|^2 + \dfrac{\beta}{2}\trans{\uvec}K\uvec
\end{align}
where $\Phi_i$ is the $i$th row of matrix $\Phi$. The first term in 
$\mathrm{Reg}(U,V)$ couples node displacements to SSM vertex displacements.  The scaling factor $\alpha$ should be
relatively large to enforce this condition.  The second term minimizes elastic energy of the nodes, favouring smaller or no 
displacements when there are additional degrees of freedom.  Through some careful manipulation, we can rewrite the above in matrix
form as
\begin{align}
    Q(U,V,s, R,t) & = \dfrac{1}{2\sigma^2}\left[N_P\trans{t}t + \trace\left(\trans{X}\diag\left(\trans{P}1\right)X\right)-2\trans{t}\trans{X}\trans{P}1\right.\notag\\
    & \qquad +s^2\;\trace\left(\trans{(Y+V)}\diag\left(P1\right)(Y+V)\right) + 2s\;\trans{t}R\trans{(Y+V)}P1\notag\\
    & \qquad \left.-2s\;\trace\left(\trans{X}\trans{P}(Y+V)\trans{R}\right)\right]+\dfrac{N_PD}{2}\log(\sigma^2)\notag\\
    & \qquad + \dfrac{\alpha}{2}\sum_{i=1}^M\|\trans{v}_i-\Phi_iU\|^2 + \dfrac{\beta}{2}\trans{\uvec}K\uvec
\end{align}
To deal with the fact that $u$ is $3J\times1$, and the the stiffness matrix $K$ involves coupling between the 
$(x,y,z)$-coordinates we re-express the system in terms of rasterized vectors:
\begin{align}
    Q(\uvec,\vvec,s, R,t) & = \dfrac{1}{2\sigma^2}\left[N_P\trans{t}t + \trans{\xvec}\diag\left(\trans{\tilde{P}}1\right)\xvec-2\trans{t}\trans{\hat{I}}\diag\left(\trans{\tilde{P}}1\right)\xvec\right.\notag\\
    & \qquad\qquad + s^2\;\trans{(\yvec+\vvec)}\diag\left(\tilde{P}1\right)(\yvec+\vvec)\notag\\
    & \qquad\qquad \left. +2s\;\trans{t}R\trans{\tilde{I}}\diag\left(\tilde{P}1\right)(\yvec+\vvec) -2s\;\trans{\xvec}\trans{\tilde{P}}\tilde{R}(\yvec+\vvec)\right]\notag\\
    & \quad +\dfrac{N_PD}{2}\log(\sigma^2)\notag\\
    & \quad + \dfrac{\alpha}{2}\left[\trans{\vvec}\vvec -2\trans{\vvec}\tilde{\Phi}\uvec + \trans{\uvec}\trans{\Phi}\Phi \uvec\right] + \dfrac{\beta}{2}\trans{\uvec}K\uvec \label{eq:Qvector}
\end{align} 
where
\begin{align*}
    \tilde{P} & = \kron(P,I_{3\times3}) = \begin{bmatrix}
                      P_{0,0}I_{3\times3} & P_{0,1}I_{3\times3} & \cdots & P_{0,N}I_{3\times3}\\
                      P_{1,0}I_{3\times3} & P_{1,1}I_{3\times3} & \cdots & P_{1,N}I_{3\times3}\\
                      \vdots & \vdots & \ddots & \vdots\\
                      P_{M,0}I_{3\times3} & P_{M,1}I_{3\times3} & \cdots & P_{M,N}I_{3\times3}\\
                  \end{bmatrix}_{3M\times3N}\\
    \tilde{\Phi} & = \kron(\Phi,I_{3\times3})\\
    \tilde{R} & = \kron(R, I_{M\times M})\\
    \trans{\hat{I}} & = \begin{bmatrix}
                    I_{3\times3} & \cdots & I_{3\times3}
                \end{bmatrix}_{3\times 3N}\\
    \trans{\tilde{I}} & = \begin{bmatrix}
                    I_{3\times3} & \cdots & I_{3\times3}
                \end{bmatrix}_{3\times 3M}
\end{align*}

\subsubsection*{Minimization for \texorpdfstring{$U$ and $V$}{U and V}}

Assume we know $(X, Y, \Phi, s, R, t)$.  We want to solve Equation \eqref{eq:Qvector} for the minimizing $U$ and $V$, 
or equivalently for $\uvec$ and $\vvec$.  First, let's remove all constant terms:
\begin{align}
    Q(\uvec,\vvec) & = \dfrac{s^2}{2\sigma^2}\trans{\vvec}\diag\left(\tilde{P}1\right)\vvec\notag\\
    & \quad + \dfrac{1}{\sigma^2}\left[\left(s^2\trans{\yvec}+s\;\trans{t}R\trans{\tilde{I}}\right)
        \diag\left(\tilde{P}1\right)-s\;\trans{\xvec}\trans{\tilde{P}}\tilde{R}\right]\vvec\notag\\
    & \quad + \dfrac{\alpha}{2}\left[\trans{\vvec}\vvec -2\trans{\vvec}\tilde{\Phi}\uvec + 
       \trans{\uvec}\trans{\Phi}\Phi \uvec\right] + \dfrac{\beta}{2}\trans{\uvec}K\uvec \label{eq:Quv}\\
       & = \dfrac{1}{2}
           \begin{bmatrix}
               \trans{\uvec} & \trans{\vvec}
           \end{bmatrix} 
           \begin{bmatrix}
               \alpha\trans{\tilde{\Phi}}\tilde{\Phi}+\beta K & -\alpha\trans{\Phi}\\
               -\alpha\Phi & \frac{s^2}{\sigma^2}\diag\left(\tilde{P}1\right)+\alpha I
           \end{bmatrix}
           \begin{bmatrix}
               \uvec\\
               \vvec
           \end{bmatrix} \notag\\
         & \quad + 
            \begin{bmatrix}
                       0 & 
                       \frac{1}{\sigma^2}\left[\left(s^2\trans{\yvec}+s\;\trans{t}R\trans{\tilde{I}}\right)
                        \diag\left(\tilde{P}1\right)-s\;\trans{\xvec}\trans{\tilde{P}}\tilde{R}\right]
            \end{bmatrix}
            \begin{bmatrix}
               \uvec\\
               \vvec
           \end{bmatrix} \notag\\           
\end{align}
Minimizing with respect to $\tilde{\uvec} = [\uvec;\vvec]$:
\begin{align}
    \di{Q}{\tilde{\uvec}} & = 
            \begin{bmatrix}
               \alpha\trans{\tilde{\Phi}}\tilde{\Phi}+\beta K & -\alpha\trans{\Phi}\\
               -\alpha\Phi & \frac{s^2}{\sigma^2}\diag\left(\tilde{P}1\right)+\alpha I
            \end{bmatrix}
            \begin{bmatrix}
               \uvec\\
               \vvec
           \end{bmatrix}\notag\\
           & \quad - \begin{bmatrix}
                       0\\
                       \frac{-1}{\sigma^2}\left[\diag\left(\tilde{P}1\right)\left(s^2\yvec+s\tilde{I}\trans{R}t\right)
                        -s\trans{\tilde{R}}\tilde{P}\xvec\right]
            \end{bmatrix}\notag\\
    \implies \quad 
        \begin{bmatrix}
            \uvec\\
            \vvec
        \end{bmatrix} & = 
            \begin{bmatrix}
               \alpha\trans{\tilde{\Phi}}\tilde{\Phi}+\beta K & -\alpha\trans{\Phi}\\
               -\alpha\Phi & \frac{s^2}{\sigma^2}\diag\left(\tilde{P}1\right)+\alpha I
            \end{bmatrix}^{-1}
            \begin{bmatrix}
                       0\\
                       \frac{-1}{\sigma^2}\left[\diag\left(\tilde{P}1\right)\left(s^2\yvec+s\tilde{I}\trans{R}t\right)
                        -s\trans{\tilde{R}}\tilde{P}\xvec\right]
            \end{bmatrix}
\end{align}

\begin{aside}
    For comparing with Siavash's work, take $R = I$, $s=1$ and $t=0$,
\begin{align}
    \begin{bmatrix}
        \uvec\\
        \vvec
    \end{bmatrix} & = 
        \begin{bmatrix}
            \alpha\trans{\tilde{\Phi}}\tilde{\Phi}+\beta K & -\alpha\trans{\Phi}\\
            -\alpha\Phi & \frac{1}{\sigma^2}\diag\left(\tilde{P}1\right)+\alpha I
        \end{bmatrix}^{-1}
        \begin{bmatrix}
                    0\\
                    \frac{-1}{\sigma^2}\left[\diag\left(\tilde{P}1\right)\yvec
                    -\tilde{P}\xvec\right]
        \end{bmatrix}
\end{align}
Re-writing Siavash's answer in matrix form:
\begin{align}
    \begin{bmatrix}
        U\\
        V
    \end{bmatrix} & = 
        \begin{bmatrix}
            \alpha\diag\left(\trace(\trans{\Phi}\Phi)\right)+\beta K & -\alpha\trans{\Phi}\\
            -\alpha\Phi & \frac{1}{\sigma^2}\diag\left(\textcolor{red}{\trace}(P1)\right)+\alpha \textcolor{red}{J}I
        \end{bmatrix}^{-1}
        \begin{bmatrix}
                    0\\
                    \frac{-1}{\sigma^2}\left[\diag\left(\textcolor{red}{\trace}(P1)\right)Y
                    -PX\right]
        \end{bmatrix}
\end{align}

\end{aside}

\subsubsection{Simplified Formulation}

Instead of keeping around both $V$ and $U$, we can simply replace $V$ with 
the expression in Equation \eqref{eq:uV}.
\begin{align}
  Q(U,s, R,t) & = \dfrac{1}{2\sigma^2}\sum_{m,n=1}^{M,N}P(y_m|x_n)\left\|x_n- R\left(y_m+\sum_{i=1}^J\phi_{mj}u_j\right)-t\right\|^2 \notag\\
  & \qquad + \dfrac{N_PD}{2}\log(\sigma^2) + \mathrm{Reg}(U) \notag\\
  \mathrm{Reg}(U) & = \dfrac{\beta}{2}\trans{\uvec}K\uvec \notag
\end{align}

\begin{aside}
    I don't know if this formulation is limiting.  We \emph{may} have more 
    flexibility in the mixed formulation, since Equation \eqref{eq:uV}
    is implemented as a soft-constraint, as opposed to here where it is a hard
    constraint.  In the end, I'm not sure if it matters, because it turns out
    that $U$ is also affected by the Gaussian distributions.
    
    The advantage of this method is that the matrix system is much smaller,
    and there is one less parameter.
\end{aside}

\noindent We can re-write this in matrix form:
\begin{align}
    Q(U, s, R,t) & = \dfrac{1}{2\sigma^2}\left[N_P\trans{t}t + \trace\left(\trans{X}\diag\left(\trans{P}1\right)X\right)-2\trans{t}\trans{X}\trans{P}1\right.\notag\\
    & \qquad +s^2\;\trace\left(\trans{(Y+\Phi U)}\diag\left(P1\right)(Y+\Phi U)\right) \notag\\
    & \qquad + 2s\;\trans{t}R\trans{(Y+ \Phi U)}P1\notag\\
    & \qquad \left.-2s\;\trace\left(\trans{X}\trans{P}(Y+\Phi U)\trans{R}\right)\right]+\dfrac{N_PD}{2}\log(\sigma^2)\notag\\
    & \qquad + \dfrac{\beta}{2}\trans{\uvec}K\uvec,
\end{align}
or using vectors,
\begin{align}
    Q(\uvec,s, R,t) & = \dfrac{1}{2\sigma^2}\left[N_P\trans{t}t + \trans{\xvec}\diag\left(\trans{\tilde{P}}1\right)\xvec-2\trans{t}\trans{\hat{I}}\diag\left(\trans{\tilde{P}}1\right)\xvec\right.\notag\\
    & \qquad\qquad + s^2\;\trans{(\yvec+\tilde{\Phi}\uvec)}\diag\left(\tilde{P}1\right)(\yvec+\tilde{\Phi}\uvec)\notag\\
    & \qquad\qquad +2s\;\trans{t}R\trans{\tilde{I}}\diag\left(\tilde{P}1\right)(\yvec+\tilde{\Phi}\uvec)\notag\\
    & \qquad\qquad \left.-2s\;\trans{\xvec}\trans{\tilde{P}}\tilde{R}(\yvec+\tilde{\Phi}\uvec)\right]\notag\\
    & \quad +\dfrac{N_PD}{2}\log(\sigma^2) + \dfrac{\beta}{2}\trans{\uvec}K\uvec \label{eq:QnoV}
\end{align} 

\subsubsection*{Minimization for \texorpdfstring{$U$}{U}}

Assume we know $(X, Y, \Phi, s, R, t)$.  We want to solve Equation 
\eqref{eq:QnoV} for the minimizing $\uvec$.  First, let's remove all constant
terms:
\begin{align}
    Q(\uvec) & = \dfrac{s^2}{2\sigma^2}\trans{\uvec}\trans{\tilde{\Phi}}\diag\left(\tilde{P}1\right)\tilde{\Phi}\uvec\notag\\
    & \quad + \dfrac{1}{\sigma^2}\left[\left(s^2\trans{\yvec}+s\;\trans{t}R\trans{\tilde{I}}\right)
        \diag\left(\tilde{P}1\right)-s\;\trans{\xvec}\trans{\tilde{P}}\tilde{R}\right]\tilde{\Phi}\uvec\notag\\
    & \quad + \dfrac{\beta}{2}\trans{\uvec}K\uvec \label{eq:Qu}
\end{align}
Differentiating with respect to $\uvec$,
\begin{align}
    \di{Q(\uvec)}{\uvec} & = \left[\dfrac{s^2}{\sigma^2}\trans{\tilde{\Phi}}\diag\left(\tilde{P}1\right)\tilde{\Phi} + \beta K\right]\uvec\notag\\
        & \quad + \dfrac{1}{\sigma^2}\left[\trans{\tilde{\Phi}}\diag\left(\tilde{P}1\right)\left(s^2\yvec+s\tilde{I}\trans{R}t\right)
        -s\trans{\tilde{\Phi}}\trans{\tilde{R}}\tilde{P}\xvec\right]\\
    \implies \quad \uvec & = -\left[s^2\trans{\tilde{\Phi}}\diag\left(\tilde{P}1\right)\tilde{\Phi} 
        + \beta \sigma^2K\right]^{-1}\left[\trans{\tilde{\Phi}}\diag\left(\tilde{P}1\right)\left(s^2\yvec+s\tilde{I}\trans{R}t\right)
        -s\trans{\tilde{\Phi}}\trans{\tilde{R}}\tilde{P}\xvec\right]
\end{align}

\begin{aside}
    Setting $s=1$, $R=I$ and $t=0$,
    \begin{align}
        \uvec & = -\left[\trans{\tilde{\Phi}}\diag\left(\tilde{P}1\right)\tilde{\Phi} 
            + \beta \sigma^2K\right]^{-1}\left[\trans{\tilde{\Phi}}\diag\left(\tilde{P}1\right)\yvec
            -\trans{\tilde{\Phi}}\tilde{P}\xvec\right]
    \end{align}
\end{aside}


\bibliographystyle{acm}
\nocite{*}
\bibliography{prostate}
 
\end{document}
